\documentclass{article}
\usepackage{spconf,amsmath,graphicx}

\title{Optimizing DTW-Based Audio-to-MIDI Alignment and Matching}

\name{Colin Raffel and Daniel P. W. Ellis}
\address{LabROSA\\
    Columbia University\\
    New York, NY}

\begin{document}

\maketitle

\begin{abstract}
Dynamic Time Warping (DTW) has proven to be an extremely effective method for both aligning and matching recordings of songs to corresponding MIDI transcriptions.
The performance of DTW-based approaches in this domain is heavily effected by system design choices, such as the representation used for the audio and MIDI data and DTW's adjustable hyperparameters.
We propose a method for optimizing the design of DTW-based alignment and matching systems.
Our technique uses Bayesian optimization to tune system design and hyperparameters over a synthetically created dataset of audio and MIDI pairs.
We then perform an exhaustive search over DTW score normalization techniques in order to determine an optimal method for reporting a reliable alignment confidence score, which is necessary for matching tasks.
Using our approach, we are able to create a DTW-based system which is conceptually simple and highly accurate at both alignment and matching.
We also verified that our system achieves high performance in a large-scale qualitative evaluation of results on real-world data.
\end{abstract}

\begin{keywords}
Dynamic Time Warping, Audio to MIDI Alignment, Sequence Retrieval, Bayesian Optimization, Hyperparameter Optimization
\end{keywords}

\section{Introduction}
\label{sec:intro}

\cite{hu2003polyphonic}

\bibliographystyle{IEEEbib}
\bibliography{refs}

\end{document}
